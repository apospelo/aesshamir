\documentclass[11pt]{article}
\usepackage[scale=.78]{geometry}

%\usepackage[margin=1in]{geometry}


\usepackage[backref]{hyperref}
\usepackage{amsmath,amsfonts,amssymb}

\newcommand\ignore[1]{}

\begin{document}

\title{Improved Protection of AES with Shamir's Secret Sharing Scheme}

\author{
Jean-S\'ebastien Coron}

\maketitle

\begin{abstract}
At CHES 2011 Goubin and Martinelli described a new countermeasure
against side-channel analysis for AES based on Shamir's secret-sharing
scheme. However their countermeasure has ${\cal O}(d^3)$ complexity
for security against $d$-th order attack, instead of ${\cal O}(d^2)$
for the Boolean masking coutermeasure.
In this paper we show a variant
with complexity ${\cal O}(d^2)$.
\end{abstract}

\section{Description}

We work in a finite field $GF(2^n)$. Let $\alpha \in GF(2^n)$, with
$\alpha \neq 0$. Given
a sensitive variable $y$, 
we consider a random
polynomial $a(x)$ of degree $\leq d$, such that $a(\alpha)=y$;
therefore the polynomial $a(x)$ has $d+1$ coefficients. The goal is to
be secure against a $d$-th order attack, or at least $d/2$.

\subsection{Addition}

Given two sensitive variables $y$ and $z$ with $a(\alpha)=y$ and
$b(\alpha)=z$, we can compute $y+z$ by computing $a(x)+b(x)$. 

\subsection{Multiplication}

To compute $y \cdot z$, we first compute the polynomial 
$$c(x)=a(x) \cdot b(x)$$
Then $c(\alpha)=a(\alpha)b(\alpha)$. However $c(x)$ is of maximum
degree $2d$ instead of $d$. Therefore we collapse some of the
coefficients of $c(x)$ to obtain $c'(x)$ in the following way. We
write:

\begin{eqnarray*}
 c(\alpha) & = & \sum\limits_{i=0}^{2d} c_i \cdot \alpha^{i}\\
& = & \sum\limits_{i=0}^{d-1} \left(c_i \cdot \alpha^i + c_{d+i+1} \cdot
 \alpha^{d+i+1}\right) + c_d \cdot \alpha^d\\
& = & \sum\limits_{i=0}^{d-1} \left(c_i + c_{d+i+1} \cdot \alpha^{d+1}
 \right) \cdot \alpha^i + c_d \cdot \alpha^d
\end{eqnarray*}

Therefore we let $c'_i=c_i+c_{d+i+1}$ and $c'_d=c_d$. We get
$c'(\alpha)=c(\alpha)$ as required.

\subsubsection{Computing $c(x)=a(x) \cdot b(x)$}

We write:

\begin{eqnarray*}
c(x) & =& \left(\sum\limits_{i=0}^d a_i x^i \right)
\left(\sum\limits_{j=0}^d b_j x^j \right) \\
& = & \sum\limits_{i=0}^d \sum\limits_{j=0}^d a_i b_j
x^{i+j}=\sum\limits_{k=0}^{2d} c_k x^k\\
%& = & \sum\limits_{k=0}^{2d} \left( \sum\limits_{i=0}^k a_i b_{k-j} \right)x^k
\end{eqnarray*}

Then the technique consists in computing the partial sums
$$ c_k=\sum\limits_{i+j=k} a_i b_j $$
in the same way as they are computed in the Ishai {\sl et al.} paper
of Crypto 2003 and in the Rivain-Prouff paper of CHES 2010. 

In the CHES 2010 paper, one must compute the product:
\begin{eqnarray*}
 c&=&\left(\sum\limits_{i=0}^d a_i  \right)
\left(\sum\limits_{j=0}^d b_j  \right) \\
& = & \sum\limits_{i=0}^d \sum\limits_{j=0}^d a_i b_j
\end{eqnarray*}
and the partial sum:
$$c_i=\sum\limits_{j=0}^d a_i b_j$$
is essentially computed by adding for every $j$ a  random
$r_{i,j}$ to both $c_i$ 
and $c_j$. 

Similary when computing the partial sum
$$ c_k=\sum\limits_{i+j=k} a_i b_{j} $$
we can add for every $i$ a random $r$ to $c_k$ and a random $r \cdot
\alpha^{k-k'}$ to $c_{k'}$ for a well chosen (varying) $k'$.  

\section{Security Proof}

The previous countermeasure is secure against $d/2$-th order masking.

\end{document}












\end{document}
